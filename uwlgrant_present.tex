\documentclass{beamer}
\usepackage{beamerthemeshadow}
\usepackage{verbatim}

\usepackage{lastpage}
\usepackage{xcolor}
\usepackage{pgf}
\usepackage{colortbl}
\usepackage{hyperref}

\newcommand{\bi}{\begin{itemize}}
\newcommand{\ei}{\end{itemize}}
\newcommand{\be}{\begin{enumerate}}
\newcommand{\ee}{\end{enumerate}}
\newcommand{\bd}{\begin{description}}
\newcommand{\ed}{\end{description}}
\newcommand{\prbf}[1]{\textbf{#1}}
\newcommand{\prit}[1]{\textit{#1}}
\newcommand{\beq}{\begin{equation}}
\newcommand{\eeq}{\end{equation}}
\newcommand{\bdm}{\begin{displaymath}}
\newcommand{\edm}{\end{displaymath}}

\newcommand{\ft}[1]{
  \frametitle{\begin{tabular}{p{4.2in}r} \textcolor{white}{#1} & \small{\insertframenumber / \inserttotalframenumber} \end{tabular}}
  \setbeamercovered{transparent=18}
}

\newcommand{\eft}[1]{
  \frametitle{\begin{tabular}{p{4in}r} \textcolor{white}{#1} & \small{\hyperlink{f:questions}{\beamergotobutton{GO BACK}}} \end{tabular}}
  \setbeamercovered{transparent=18}
}

\newcommand{\stepinv}{\setbeamercovered{invisible}}
\newcommand{\stopinv}{\setbeamercovered{transparent=18}}
\newcommand{\uncoverinv}[1]
{
  \setbeamercovered{invisible}
  \uncover<+->{#1}
  \setbeamercovered{transparent=18}
}
\newcommand{\ans}[1]{\textcolor{blue}{#1}}
\newcommand{\ansinv}[1]
{
  \setbeamercovered{invisible}
  \uncover<+->{\textcolor{blue}{#1}}
  \setbeamercovered{transparent=18}
}
\newcommand{\setinv}{\setbeamercovered{invisible}}
\newcommand{\setvis}{\setbeamercovered{transparent=18}}
\newcommand{\centerpic}[2]
{
  \begin{center}
  \includegraphics[#1]{#2}
  \end{center}
}
\newcommand{\h}[1]{\hat{#1}}
\newcommand{\ds}{\displaystyle}

%\definecolor{light}{rgb}{1.0,0.33,0.33}
\definecolor{light}{rgb}{1.0,0.5,0.5}
\newcommand{\hl}[1]{\alt<#1>{\rowcolor{light}\hspace*{-2.1pt}} {\hspace*{-2.1pt}} }

\definecolor{mycolor}{rgb}{0.6,0.0,0.0}
\usecolortheme[named=mycolor]{structure}

\title[Labor Markets and Adaptive Expectations]{Labor Markets and Adaptive Expectations:}
\subtitle{Estimating a New Keynesian Model with Learning and Unemployment}
\author[James Murray, Department of Economics]{James Murray\\Department of Economics\\University of Wisconsin - La Crosse}
\date{November 16, 2011}

\begin{document}

\frame{\titlepage \setcounter{framenumber}{0}}

\section{Purpose and Outcomes}
\frame
{
  \ft{Research Project Purpose and Outcomes}
  \uncover<+->{
  \begin{block}{Purpose}
  Estimate the effects adaptive expectations have on unemployment dynamics in the United States.
  \end{block}
  }

  \uncover<+->{
  \begin{block}{Outcome: Answers to the following questions}
    \be
    \item<+-> Do expectations influence how long unemployment lasts, and how severe the unemployment problem becomes?
    \item<+-> Do expectations help explain why some unemployment episodes are brief, while others are long?
    \item<+-> Can government policy effectively reduce unemployment?  Under what conditions is this more likely?   
    \item<+-> What economic shocks are responsible for U.S. business cycle fluctuations?  
    \ee
  \end{block}
  }
}

\section{Expectations}
\subsection{Example}
\frame
{
  \ft{Expectations}
  \uncover<+->{
  \begin{block}{Example of Destabilizing Effect of Expectations}
    \be
    \item<+-> Suppose consumers expect unemployment rate to rise.
    \item<+-> As a precautionary measure, many consumers save more, spend less.
    \item<+-> Lower spending leads to lower sales, revenues, profits.
    \item<+-> Businesses cut production, cut employees
    \item<+-> Unemployment does rise (expectations self-fulfilling, destabilizing)
    \ee
  \end{block}
  }

  \uncover<+->{
  \begin{block}{Expectation Frameworks}
    \bi
    \item<.-> I will use least-squares learning (adaptive expectations)
    \item<.-> Rational expectations is the typical framework in the macroeconomics literature.
    \ei
  \end{block}
  }
}

\subsection{Learning Literature}
\frame
{
  \ft{Literature: Least Squares Learning}
  \small{
  \uncover<+->{
  \begin{block}{Prolonged Inflation and Volatility}
    \bi
    \item<.-> Learning can lead to prolonged periods of inflation (Orphanides and Williams, 2005a)
    \item<.-> Learning explains the run-up of inflation and volatility in the 1970s (Orphanides and Williams, 2005b), ...
    \item<.-> ... and subsequent stability starting in mid-1980s (Primiceri, 2006) 
    \ei
  \end{block}
  }

  \uncover<+->{
  \begin{block}{Business Cycle Dynamics}
    \bi
    \item<.-> Learning explains prolonged economic downturns and larger bubbles (Eusepi and Preston, 2011) 
    \item<.-> Learning explains different-sized economic downturns and upswings (Milani, 2009)
    \item<.-> Expectations significant cause for business cycles (Milani, 2011)
    \ei
  \end{block}
  }
}
}

\section{Unemployment}
\frame
{
  \ft{Need for More Research: Unemployment}
  \uncover<+->{
  \begin{block}{Aforementioned papers ignore unemployment}
    \bi
    \item<+-> Expectations about unemployment influence economy.
    \item<+-> Including data on unemployment can affect results (Gertler, Sala, and Trigari, 2008)
    \item<+-> Current (very prolonged) unemployment situation.
    \ei
  \end{block}
  }

  \uncover<+->{
  \begin{block}{Existing Unemployment (Macroeconometrics) Literature}
    \bi
    \item<+-> Blanchard and Gali (2010): Pioneer work modeling unemployment in macroeconometrics literature.
    \item<+-> Existing work focuses on monetary policy (Thomas, 2008;  Blanchard and Gali, 2010)
    \item<+-> Existing work does not explore implications for business cycles.
    \item<+-> Existing work assumes \textit{rational expectations}.
    \ei
  \end{block}
  }
}

\section{}
\subsection{Research Project Outcomes}
\frame
{
  \ft{Research Project Outcomes}

  \footnotesize{
  \uncover<+->{
  \begin{block}{Outcomes: Answers to the following questions}
    \be
    \item Do expectations influence how long unemployment lasts, and how deep the unemployment problem becomes?
    \item Do expectations help explain why some unemployment episodes are brief, while others are long?
    \item Can government policy effectively reduce unemployment?  Under what conditions is this more likely?   
    \item What economic shocks are responsible for U.S. business cycle fluctuations?  
    \ee
  \end{block}
  }

  \uncover<+->{
  \begin{block}{Personal Growth}
    \bi
    \item This is related to my 2008 Ph.D. dissertation.
    \item Part of a research program I am starting to build.
    \item Likely to lead to more projects in this field.  Eg: Expectations about future government actions and unemployment.
    \ei
  \end{block}
  }
  }
}


\end{document}

